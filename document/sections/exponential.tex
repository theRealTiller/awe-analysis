\section{Exponential Mapping and its First Derivative}


The exponential mapping $ \Exp $ of the surface $ M $
is understood as a family
\begin{align}
    \left( \Exp( p; - ) : p \in M \right)
\end{align}
of mappings $ \Exp( p; - ) \colon \mathrm{T}_p M \to M $.
The point $ \Exp( p; v ) $ is defined as the endpoint
of the geodesic segment of length $ \abs{ v } $ 
that starts at $ p $ in the direction of $ v $.


\subsection{Construction}


Given a point $ p_0 $ on $ M $ and
a tangent vector $ v_0 $ of $ M $ at $ p_0 $
one defines
\begin{subequations}
    \label{def:gem}
    \begin{align}
        \label{def:gem:a}
        \Exp( p_0; v_0 )  :=
        p_0  +  \int^1_0 v( t ) \intd{ t } ,
    \end{align}
    where $ t \mapsto p(t) $ is a geodesic satisfying the initial conditions
    \begin{align}
        p( 0 )  &\overset{ ! }{ = }   p_0 ,
        \label{def:gem:b}\\
        v( 0 )  &\overset{ ! }{ = }   v_0 .
        \label{def:gem:c}
    \end{align}
\end{subequations}
Thus, the value $ \Exp( p_0; v_0 ) $
is determined by the Equations \eqref{def:cf} and \eqref{def:gem}.


In order to obtain a formula for the first derivative $ \diff{}{ \Exp } $
of the exponential mapping one makes use
of variational derivatives $ \vardiff $ of curves.
Applying the variational derivative $ \vardiff $
to Equation~\eqref{def:cf} according to
\begin{align}
    \vardiff \left(
    \frac{ \grad{F} }{ \skaq{ \grad{F} } } \mult \diff{2}{F}( v )
    +
    \dot{ v }
    \right)
    =
    0
    \label{def:cf_vd}
\end{align}
results in a differential equation for $ \vardiff $ as follows.
One calculates the variational derivatives
\begin{subequations}
    \begin{align}
    \vardiff \left( \frac{ \grad{ F } }{ \skaq{ \grad{ F } } } \right)
    &=
    \frac{ \diff{}{ \grad{ F } }( \vardiff p ) }{ \skaq{ \grad{ F } } }  -
    2 \mult \grad{ F } \mult
    \frac{ \diff{ 2 }{ F }( \vardiff p, \grad{ F } ) }{ \skaq{ \grad{ F } }^2 }
    \\
    \vardiff\left( \diff{ 2 }{ F }( v ) \right)
    &=
    \diff{ 3 }{ F }( v, v, \vardiff p )  +
    2 \mult \diff{ 2 }{ F }( v, \vardiff v )
    \end{align}
\end{subequations}
and applies the product rule in Equation~\eqref{def:cf_vd}.
This results in the differential equation
\begin{subequations}
    \label{def:cf_vd_dgl}
    \begin{align}
        C_0( v, \vardiff p )  +  C_1( v, \vardiff v )  +  \vardiff \dot{ v }
        &=
        0
        \label{def:cf_vd_dgl:a}
    \end{align}
    with the coefficients
    \begin{align}
        C_0( v, \vardiff p )
        &=
        \frac{ \grad{F} }{ \skaq{ \grad{F} } } \mult
        \left[  \diff{3}{F}( v, v, \vardiff p )  -
        2 \mult \diff{2}{F}( \vardiff p, \grad{F} ) \mult
        \frac{ \diff{2}{F}( v ) }{ \skaq{\grad{F}} }  \right]  +
        \diff{}{\grad{F}}( \vardiff p ) \mult
        \frac{ \diff{2}{F}( v ) }{ \skaq{\grad{F}} } ,
        \label{def:cf_vd_dgl:b}
        \\
        C_1( v, \vardiff v )
        &=
        2 \mult \frac{ \grad{F} }{ \skaq{ \grad{F} } }  \mult
        \diff{2}{F}( v, \vardiff v ) .
        \label{def:cf_vd_dgl:c}
    \end{align}
\end{subequations}


Then the partial derivative of the exponential mapping
in the variable $ v_0 $ is explained by
\begin{subequations}
    \label{def:dgem}
    \begin{align}
        \diff{}{ \Exp }( p_0; v_0; u_0 )  =
        \int^1_0 \vardiff v( t ) \intd{ t }
        \label{def:dgem:a}
    \end{align}
    where $ \vardiff p $ is the curve,
    with $ \vardiff v  :=  ( \vardiff p)^\bullet $,
    that satisfies the differential equation \eqref{def:cf_vd_dgl}
    and determined by the initial values
    \begin{align}
        \vardiff p( 0 )  &\overset{ ! }{ = }  0,
        \label{def:dgem:c}\\
        \vardiff v( 0 )  &\overset{ ! }{ = }  u_0 .
        \label{def:dgem:d}
    \end{align}
\end{subequations}



The evolution equation \eqref{def:cf_vd_dgl}
can also be written in the form
\begin{subequations}
    \begin{align}
        \vardiff \dot{ v }  +
        \darg{F}( v ) \mult
        [   \skap{ A_0( v ) }{ \vardiff p }  + 
            \skap{ A_1( v ) }{ \vardiff v }     ]  +
        B( v ) \mult \vardiff p
        &=  0
    \end{align}
    with the expressions
    \begin{align}
        A_0( v )  &=
        \diff{2}{\grad{F}}( v )  -
        2 \mult \diff{}{\grad{F}}( \darg{F} ) \mult \diff{2}{F}( v ) ,
        \\
        A_1( v )  &=
        2 \mult \diff{}{\grad{F}}( v ) ,
        \\
        B( v )  &=
        \nabla^2F \mult \frac{ \diff{2}{F}(v ) }{ \skaq{ \grad{F} } } .
    \end{align}
\end{subequations}


\subsection{Transversal Directional Derivatives}



Consider the case where $ u_0 $ is of the form
\begin{align}
    u_0  =  \alpha_0 \mult \grad{F}( p_0 ) \times v_0 .
\end{align}
In the following it is demonstrated,
that associated geodesic variation is of the form
\begin{align}
    \vardiff p  =  \alpha \mult \grad{ F } \times v ,
    \label{def:ta}
\end{align}
with a scalar function $ \alpha $.



The first derivative of \eqref{def:ta} is
\begin{align}
    \vardiff v  =
    \dot{ \alpha } \mult \grad{ F } \times v  +
    \alpha \mult \diff{}{ \grad{ F } }( v ) \times v
    \label{eq:tad}
\end{align}
and the second derivative is
% \begin{align}
%     \vardiff \dot{ v }  &=
%     \ddot{ \alpha } \mult \grad{ F } \times v  +  \dots\nonumber\\
%     &\hphantom{{}={}}\dot{ \alpha } \mult 2 \mult \diff{}{ \grad{ F } }( v ) \times v  +  \dots\nonumber\\
%     &\hphantom{{}={}}\alpha \mult \left[ \diff{}{ \grad{ F } }( v ) \times \dot{ v }  +
%     \diff{}{ \grad{ F } }( \dot{ v } ) \times v  +  
%     \diff{ 2 }{ \grad{ F } }( v ) \times v \right] .
%     \label{eq:tadd}
% \end{align}
\begin{align}
    \vardiff \dot{ v }  &=
    \ddot{ \alpha } \mult \grad{ F } \times v  +
    2 \mult \dot{ \alpha } \mult \diff{}{ \grad{ F } }( v ) \times v  +
    \alpha \mult \left[ \diff{}{ \grad{ F } }( v ) \times \dot{ v }  +
    \diff{}{ \grad{ F } }( \dot{ v } ) \times v  +  
    \diff{ 2 }{ \grad{ F } }( v ) \times v \right] .
    \label{eq:tadd}
\end{align}


First note, that the initial condition \eqref{def:dgem:c} translates to
\begin{align}
    \alpha( 0 )  \overset{!}{=}  0 .
\end{align}
Further, the initial condition \eqref{def:dgem:c}
and Equation \eqref{def:dgem:d} yield
\begin{align}
    \vardiff v( 0 )   \overset{!}{=}
    \dot{ \alpha }( 0 ) \mult \grad{F}( p( 0 ) ) \times v( 0 )  =  u_0 .
\end{align}
This results in the relation
\begin{align}
    \dot{ \alpha }_0   =
    \frac{ \abs{ u_0 } }{ \abs{ \grad{F}[ p_0 ] } } .
\end{align}


In order to proof the claim above,
the approach is to take the scalar product
of the right hand sides of \eqref{def:cf_vd_dgl} und \eqref{eq:tadd}
with each $ v $, $ \grad{F} $ and $ \grad{F} \times v $.


Starting with $ v $ one calculates
\begin{align}
    \skap{ v }{ \vardiff \dot{ v } }  &\overset{\eqref{def:cf_vd_dgl}}{=}
    -\alpha \mult \skap{ v }{ \diff{}{\grad{F}}( \grad{F} \times v ) }  \mult
    \frac{ \diff{2}{F}( v ) }{ \skaq{ \grad{F} } }
    \label{eq:v_da_1}
\end{align}
and
\begin{align}
    \skap{ v }{ \vardiff \dot{ v } }  &\overset{\eqref{eq:tadd}}{=}
    -\alpha \mult \skap{ v }{ \diff{}{\grad{F}}( v ) \times \grad{F} }  \mult
    \frac{ \diff{2}{F}( v ) }{ \skaq{ \grad{ F } } } .
    \label{eq:v_da_2}
\end{align}
However, due to
\begin{align}
    \skap{ v }{ \diff{}{\grad{F}}( v ) \times \grad{F} }  =
    \skap{ \grad{F} \times v }{ \diff{}{\grad{F}}( v ) }  =
    \diff{2}{F}( v, \grad{F} \times v )
\end{align}
it follows that the right hand sides
of \eqref{eq:v_da_1} and \eqref{eq:v_da_2} are equal.
Thus, the $ \grad{F} $-komponent of equation \eqref{def:cf_vd_dgl}
is always fulfilled for the ansatz \eqref{eq:ta}.


Next one examines the $ \grad{F} $-component of \eqref{def:cf_vd_dgl}.
Calculating the scalar product of $ \grad{F} $ with $ C_0( v, \vardiff p ) $
yields
\begin{subequations}
    \label{vdgk:1}
    \begin{align}
        \skap{ \grad{F} }{ - \darg{F} \mult
        \diff{3}{F}( v, v, \vardiff p ) }
        &=
        - \alpha \mult \diff{3}{F}( v, v, \grad{F} \times v ) ,
        \label{vdgk:1:a}
        \\
        \skap{ \grad{F} }
        { 2 \mult \darg{F} \mult \diff{2}{F}( v ) \mult
        \diff{2}{F}( \vardiff p, \darg{F} ) }
        &=
        2 \mult \alpha \mult \diff{2}{F}( v ) \mult
        \diff{2}{F}( \grad{F}, \darg{F} \times v ) ,
        \label{vdgk:1:c}
        \\
        \skap{ \grad{F} }
        { -\diff{}{\grad{F}}( \vardiff p ) \mult \diff{2}{F}( v ) / \skaq{ \grad{F} } }
        &=
        - \alpha \mult \diff{2}{F}( v ) \mult
        \diff{2}{F}( \grad{F}, \darg{F} \times v ) .
        \label{vdgk:1:d}
    \end{align}
\end{subequations}
Taking the scalar product of $ \grad{F} $
with $ C_1( v, \vardiff v ) $ one obtains
\begin{align}
        \skap{ \grad{F} }{ -2 \mult \darg{F} \mult \diff{2}{F}( v, \vardiff p ) }
        &=
        - 2 \mult \dot{ \alpha }  \mult  \diff{2}{F}( v, \grad{F} \times v )
        \label{vdgk:1:b}
\end{align}
from
\begin{align*}
    \skap{ \grad{F} }{ -2 \mult \darg{F} \mult \diff{2}{F}( v, \vardiff v ) }
    &=
    - 2 \left[ \dot{ \alpha }  \mult  \diff{2}{F}( v, \grad{F} \times v )
    +
    \alpha  \mult \diff{2}{F}( v, \diff{}{\grad{F}}( v ) \times v ) \right]
\end{align*}
and
\begin{align*}
    \diff{2}{F}( v, \diff{}{\grad{F}}( v ) \times v )
    =
    \skap{ \diff{}{\grad{F}}( v ) }{ \diff{}{\grad{F}}( v ) \times v }
    =
    0 .
\end{align*}
Taking the scalar product of $ \grad{F} $
with the right hand side of \eqref{eq:tadd} yields the terms
\begin{subequations}
    \label{vdgk:2}
    \begin{align}
        \skap{ \grad{F} }
        { 2 \mult \dot{ \alpha } \mult \diff{}{\grad{F}}( v ) \times v }
        &=
        - 2 \mult \dot{ \alpha } \mult \diff{2}{F}( v, \grad{F} \times v ) ,
        \label{vdgk:2:a}
        \\
        \skap{ \grad{F} }
        { \alpha \mult \diff{}{ \grad{F} }( \dot{ v } ) \times v }
        &=
        \alpha \mult \diff{2}{F}( v ) \mult
        \diff{2}{F}( \grad{F}, \darg{F} \times v ) ,
        \label{vdgk:2:b}
        \\
        \skap{ \grad{F} }
        { \alpha \mult \diff{2}{\grad{F}}( v ) \times v }
        &=
        - \alpha \mult \diff{3}{F}( v, v, \grad{F} \times v ) .
        \label{vdgk:2:c}
    \end{align}
\end{subequations}
Equation \eqref{vdgk:2:b} was obtained from
\begin{align*}
    \skap{ \grad{F} }
    { 2 \mult \dot{ \alpha } \mult \diff{}{\grad{F}}( v ) \times v }   =
    - 2 \mult \dot{ \alpha } \mult
    \skap{ \grad{F} \times v }{ \diff{}{ \grad{F}( v ) } }   =
    - 2 \mult \dot{ \alpha } \mult \diff{2}{F}( v, \grad{F} \times v ) .
\end{align*}
Now, one recognizes, that $ \eqref{vdgk:1:a}  =  \eqref{vdgk:2:c} $,
$ \eqref{vdgk:1:b}  =  \eqref{vdgk:2:a} $ and
$ \eqref{vdgk:1:c} + \eqref{vdgk:1:d}  =  \eqref{vdgk:2:b} $.
Thus, the $ \grad{F} \times v $-komponent of equation \eqref{def:cf_vd_dgl}
is always fulfilled for the ansatz \eqref{eq:ta}.



One calculates the expressions
\begin{subequations}
    \begin{align}
        \skap{ \grad{ F } \times v }
        { \ddot{ \alpha } \mult \grad{ F } \times v }
        &=
        \ddot{ \alpha } \mult \skaq{ \grad{ F } } \skaq{ v } ,
        \\
        \skap{ \grad{ F } \times v }
        { 2 \mult \dot{ \alpha } \mult \diff{}{ \grad{ F } }( v ) \times v }
        &=
        2 \mult \dot{ \alpha } \mult \skaq{ v } \mult \diff{2}{ F }( v, \grad{ F } ) ,
        \\
        \skap{ \grad{ F } \times v }
        { \alpha \mult \diff{}{ \grad{ F } }( v ) \times \dot{ v } }
        &=
        \alpha \mult  ( \diff{2}{ F }( v ) )^2 ,
        \\
        \skap{ \grad{ F } \times v }
        { \alpha \mult \diff{}{ \grad{ F } }( \dot{ v } ) \times v }
        &=
        - \alpha \mult \skaq{ v } \mult \diff{2}{ F }( \grad{ F } ) \mult
        \diff{2}{ F }( v ) / \skaq{ \grad{ F } } ,
        \\
        \skap{ \grad{ F } \times v }
        { \alpha \mult \diff{2}{ \grad{ F } }( v ) \times v }
        &=
        \alpha \mult \skaq{ v } \mult \diff{3}{ F }( v, v, \grad{ F } ) .
    \end{align}
\end{subequations}
Equation \eqref{def:cf_vd_dgl} yields
\begin{align}
    \skap{ \grad{ F } \times v }{ \vardiff \dot{ v } }
    &=
    -\alpha \mult \diff{2}{ F }( \grad{ F } \times v ) \mult
    \diff{2}{ F }( v ) / \skaq{ \grad{ F } } .
\end{align}
The equation can be cast in the form
\begin{subequations}
    \begin{align}
        c_0 \mult \alpha  +
        c_1 \mult \dot{ \alpha }  +
        \ddot{ \alpha }
        =
        0
    \end{align}
    with the coefficients
    \begin{align}
        c_0
        &=
        \frac{ \diff{3}{F}( v, v, \grad{F} ) }{ \skaq{\grad{F}} }  +
        \diff{2}{F}( v ) \mult
        \left[ \frac{ 1 }{ \skaq{v} } \mult
        \left[ \frac{ \diff{2}{F}( v ) }{ \skaq{\grad{F}} }  +
        \frac{ \diff{2}{F}( \grad{F} \times v ) }{ \skaq{\grad{F}}^2 } \right]
        -  \frac{ \diff{2}{F}( \grad{F} ) }{ \skaq{\grad{F}}^2 } \right] ,
        \\
        c_1
        &=
        2 \mult \frac{ \diff{2}{F}( v, \grad{F} ) }{ \skaq{\grad{F}} } .
    \end{align}
\end{subequations}
The coefficients can be rewritten a little more compactly as
\begin{subequations}
    \begin{align}
        c_0   &=
        \diff{3}{F}( v, v, \darg{F} )  +
        \diff{2}{F}( v )
        \left[ \frac{ \frac{ \diff{2}{F}( v ) }{ \skaq{\grad{F}} }  +
        \diff{2}{F}( \darg{F} \times v ) }{ \skaq{v} }  -
        \diff{2}{F}( \darg{F} ) \right] ,
        \\
        c_1   &=   2 \mult \diff{2}{F}( v, \darg{F} ) .
    \end{align}
\end{subequations}



% Transversals property
% \begin{align}
%     \skap{ \vardiff p }{ v }^\bullet  =
%     \skap{ \vardiff v }{ v }  +  \skap{ \vardiff p }{ \dot{ v } }  =
%     \vardiff( \tfrac12 \skaq{ v } )  +  \skap{ \vardiff p }{ \dot{ v } }
% \end{align}