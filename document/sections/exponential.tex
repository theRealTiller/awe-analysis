\section{The Exponential Mapping and its First Derivative}



\subsection{The Construction}


The exponential mapping associated to $ F $ is defined as
\begin{subequations}
    \label{def:gem}
    \begin{align}
        \label{def:gem:a}
        \operatorname{Exp}( F; p_0; v_0 )  :=
        p_0  +  \int^1_0 v( t ) \intd{ t } ,
    \end{align}
    where $ v $ is a geodesic satisfying the initial conditions
    \begin{align}
        p( 0 )  &\overset{ ! }{ = }   p_0 ,
        \label{def:gem:b}\\
        v( 0 )  &\overset{ ! }{ = }   v_0 ,
        \label{def:gem:c}
    \end{align}
\end{subequations}
where $ p_0 $ is a point on $ M $ and
$ v_0 $ is a vector tangential to $ M $ at $ p_0 $.
Thus, the value $ \operatorname{Exp}(F;p_0;v_0) $
is determined by the Equations \eqref{def:cf},
\eqref{def:gem:b} and \eqref{def:gem:c}.


The partial derivative of the exponential mapping
in the variable $ v_0 $ is explained by
\begin{subequations}
    \label{def:dgem}
    \begin{align}
        \diff{}{ \operatorname{Exp} }( F; p_0; v_0; u_0 )  =
        \int^1_0 \delta v( t ) \intd{ t }
        \label{def:dgem:a}
    \end{align}
    where $ \delta p $ is the curve
    with $ \delta v  :=  ( \delta p)^\bullet $
    determined by the initial values
    \begin{align}
        \delta p( 0 )  &\overset{ ! }{ = }  0,
        \label{def:dgem:c}\\
        \delta v( 0 )  &\overset{ ! }{ = }  u_0
        \label{def:dgem:d}
    \end{align}
    and the evolution equation
    \begin{align}
        \delta \dot{ v }   &=
        A( v, \delta p )  +  B( v, \delta v )
        \label{def:dgem:b}
    \end{align}
    with
    \begin{align}
        &\begin{aligned}
            A( v, \delta p )   &=
            - \frac{ \grad{F} }{ \skaq{ \grad{F} } }
            \left[  \diff{3}{F}( v, v, \delta p )  -
            2 \mult \diff{2}{F}( \delta p, \grad{F} ) \mult
            \frac{ \diff{2}{F}( v ) }{ \skaq{\grad{F}} }  \right]  +  \dots\\
            &\hphantom{={}} - \diff{}{\grad{F}}( \delta p ) \mult
            \frac{ \diff{2}{F}( v ) }{ \skaq{\grad{F}} } ,
        \end{aligned}
        \label{def:dgem:e}\\
        &B( v, \delta v )   =
        - 2 \mult \frac{ \grad{F} }{ \skaq{ \grad{F} } }  \mult
        \diff{2}{F}( v, \delta v ) .
        \label{def:dgem:f}
    \end{align}
\end{subequations}


The right hand side of Equation \eqref{def:dgem:b} is obtained
by applying the variation operator $ \delta $
to both sides of Equation \eqref{def:cf}, which yields
\begin{align}
    \delta \dot{ v }  =
    \delta \left(
        -\frac{ \grad{F} }{ \skaq{ \grad{F} } }  \mult
        \diff{2}{F}( v, \delta p ) \right) .
\end{align}
Here one calculates
\begin{subequations}
    \begin{align}
    \delta \left( \frac{ \grad{ F } }{ \skaq{ \grad{ F } } } \right)  &=
    \frac{ \diff{}{ \grad{ F } }( \delta p ) }{ \skaq{ \grad{ F } } }  -
    2 \mult \grad{ F } \mult \frac{ \diff{ 2 }{ F }( \delta p, \grad{ F } ) }{ \skaq{ \grad{ F } }^2 }\\
    \delta\left( \diff{ 2 }{ F }( v ) \right)  &=
    \diff{ 3 }{ F }( v, v, \delta p )  +
    2 \mult \diff{ 2 }{ F }( v, \delta v )
    \end{align}
\end{subequations}
and obtains \eqref{def:dgem:b} from the product rule.



Another equivalent form of the evolution equation \eqref{def:dgem:b} is
\begin{subequations}
    \begin{align}
        \delta \dot{ v }   &=
        \darg{F}( v ) \mult
        [   \skap{ C_1( v ) }{ \delta p }  + 
            \skap{ C_2( v ) }{ \delta v }     ]  +
        C_3( v, \delta p )
    \end{align}
    with
    \begin{align}
        C_1( v )  &=
        -\diff{2}{\grad{F}}( v )  +
        2 \mult \diff{}{\grad{F}}( \darg{F} ) \mult \diff{2}{F}( v )  \\
        C_2( v )  &=
        -2 \mult \diff{}{\grad{F}}( v )  \\
        C_3( v, \delta p )  &=
        - \diff{}{\grad{F}}( \delta p ) \mult \frac{ \diff{2}{F}(v ) }{ \skaq{ \grad{F} } }
    \end{align}
\end{subequations}


\subsection{Transversal Directional Derivatives}



Consider the case where $ u_0 $ is of the form
\begin{align}
    u_0  =  \alpha_0 \mult \grad{F}( p_0 ) \times v_0 .
\end{align}
In the following it is demonstrated,
that associated geodesic variation is of the form
\begin{align}
    \delta p  =  \alpha \mult \grad{ F } \times v ,
    \label{def:ta}
\end{align}
with a scalar function $ \alpha $.



The first derivative of \eqref{def:ta} is
\begin{align}
    \delta v  =
    \dot{ \alpha } \mult \grad{ F } \times v  +
    \alpha \mult \diff{}{ \grad{ F } }( v ) \times v
    \label{eq:tad}
\end{align}
and the second derivative is
\begin{align}
    \delta \dot{ v }  &=
    \ddot{ \alpha } \mult \grad{ F } \times v  +  \dots\nonumber\\
    &\hphantom{{}={}}\dot{ \alpha } \mult 2 \diff{}{ \grad{ F } }( v ) \times v  +  \dots\nonumber\\
    &\hphantom{{}={}}\alpha \mult \left[ \diff{}{ \grad{ F } }( v ) \times \dot{ v }  +
    \diff{}{ \grad{ F } }( \dot{ v } ) \times v  +  
    \diff{ 2 }{ \grad{ F } }( v ) \times v \right] .
    \label{eq:tadd}
\end{align}


First note, that the initial condition \eqref{def:dgem:c} translates to
\begin{align}
    \alpha( 0 )  \overset{!}{=}  0 .
\end{align}
Further, the initial condition \eqref{def:dgem:c}
and Equation \eqref{def:dgem:d} yield
\begin{align}
    \delta v( 0 )   \overset{!}{=}
    \dot{ \alpha }( 0 ) \mult \grad{F}( p( 0 ) ) \times v( 0 )  =  u_0 .
\end{align}
This results in the relation
\begin{align}
    \dot{ \alpha }_0   =
    \frac{ \abs{ u_0 } }{ \abs{ \grad{F}[ p_0 ] } } .
\end{align}


In order to proof the claim above,
the approach is to take the scalar product
of the right hand sides of \eqref{def:dgem:b} und \eqref{eq:tadd}
with each $ v $, $ \grad{F} $ and $ \grad{F} \times v $.


Starting with $ v $ one calculates
\begin{align}
    \skap{ v }{ \delta \dot{ v } }  &\overset{\eqref{def:dgem:b}}{=}
    -\alpha \mult \skap{ v }{ \diff{}{\grad{F}}( \grad{F} \times v ) }  \mult
    \frac{ \diff{2}{F}( v ) }{ \skaq{ \grad{F} } }
    \label{eq:v_da_1}
\end{align}
and
\begin{align}
    \skap{ v }{ \delta \dot{ v } }  &\overset{\eqref{eq:tadd}}{=}
    -\alpha \mult \skap{ v }{ \diff{}{\grad{F}}( v ) \times \grad{F} }  \mult
    \frac{ \diff{2}{F}( v ) }{ \skaq{ \grad{ F } } } .
    \label{eq:v_da_2}
\end{align}
However, due to
\begin{align}
    \skap{ v }{ \diff{}{\grad{F}}( v ) \times \grad{F} }  =
    \skap{ v }{ \diff{}{\grad{F}}( \grad{F} \times v ) }  =
    \diff{2}{F}( v, \grad{F} \times v )
\end{align}
it follows that the right hand sides
of \eqref{eq:v_da_1} and \eqref{eq:v_da_2} are equal.



The $ \grad{F} $-component of \eqref{def:dgem:b}
for the transversal case \eqref{eq:tadd}
can be calculated to be redundant.
First one calculates the scalar product of $ \grad{F} $
with the right hand side of \eqref{def:dgem:b} term by term.
First
\begin{subequations}
    \label{vdgk:1}
    \begin{align}
        \skap{ \grad{F} }{ - \darg{F} \mult
        \diff{3}{F}( v, v, \delta p ) }   =
        - \alpha \mult \diff{3}{F}( v, v, \grad{F} \times v ) ,
        \label{vdgk:1:a}
    \end{align}
    second
    \begin{align}
        \skap{ \grad{F} }{ -2 \darg{F} \mult \diff{2}{F}( v, \delta p ) }   &=
        - 2 ( \dot{ \alpha }  \mult  \diff{2}{F}( v, \grad{F} \times v )  +
        \alpha  \mult \diff{2}{F}( v, \diff{}{\grad{F}}( v ) \times v ) )  \nonumber\\
        &=   - 2 \dot{ \alpha }  \mult  \diff{2}{F}( v, \grad{F} \times v )
        \label{vdgk:1:b}
    \end{align}
    due to
    \begin{align*}
        \diff{2}{F}( v, \diff{}{\grad{F}}( v ) \times v )  =
        \skap{ \diff{}{\grad{F}}( v ) }{ \diff{}{\grad{F}}( v ) \times v }  =  0 ,
    \end{align*}
    third
    \begin{align}
        \skap{ \grad{F} }
        { 2 \mult \darg{F} \mult \diff{2}{F}( v ) \mult
        \diff{2}{F}( \delta p, \darg{F} ) }   &=
        2 \alpha \mult \diff{2}{F}( v ) \mult
        \diff{2}{F}( \grad{F}, \darg{F} \times v )
        \label{vdgk:1:c}
    \end{align}
    and, fourth,
    \begin{align}
        \skap{ \grad{F} }
        { -\diff{}{\grad{F}}( \delta p ) \mult \diff{2}{F}( v ) / \skaq{ \grad{F} } }
        =   - \alpha \mult \diff{2}{F}( v ) \mult
        \diff{2}{F}( \grad{F}, \darg{F} \times v ) .
        \label{vdgk:1:d}
    \end{align}
\end{subequations}
Evaluating the scalar product of $ \grad{F} $
with the right hand side of \eqref{eq:tadd}
term by term yields
\begin{subequations}
    \label{vdgk:2}
    \begin{align}
        \skap{ \grad{F} }
        { 2 \dot{ \alpha } \mult \diff{}{\grad{F}}( v ) \times v }   =
        - 2 \dot{ \alpha } \mult
        \skap{ \grad{F} \times v }{ \diff{}{ \grad{F}( v ) } }   =
        - 2 \dot{ \alpha } \mult \diff{2}{F}( v, \grad{F} \times v ) ,
        \label{vdgk:2:a}
    \end{align}
    second
    \begin{align}
        \skap{ \grad{F} }
        { \alpha \mult \diff{}{ \grad{F} }( \dot{ v } ) \times v }   &=
        \alpha \mult \diff{2}{F}( v ) \mult
        \diff{2}{F}( \grad{F}, \darg{F} \times v )
        \label{vdgk:2:b}
    \end{align}
    and, third,
    \begin{align}
        \skap{ \grad{F} }
        { \alpha \mult \diff{2}{\grad{F}}( v ) \times v }   &=
        - \alpha \mult \diff{3}{F}( v, v, \grad{F} \times v ) .
        \label{vdgk:2:c}
    \end{align}
\end{subequations}
Now, one recognizes, that $ \eqref{vdgk:1:a}  =  \eqref{vdgk:2:c} $,
$ \eqref{vdgk:1:b}  =  \eqref{vdgk:2:a} $ and
$ \eqref{vdgk:1:c} + \eqref{vdgk:1:d}  =  \eqref{vdgk:2:b} $.
Thus, the $ \grad{F} $-component of $ \eqref{def:dgem:b} = \eqref{eq:tadd} $
is trivially fulfilled for $ \delta p  =  \alpha \mult \grad{F} \times v $.




One calculates the expressions
\begin{subequations}
    \begin{align}
        \skap{ \grad{ F } \times v }{ \grad{ F } \times v }  &=
        \skaq{ \grad{ F } } \skaq{ v }\\
        \skap{ \grad{ F } \times v }{ \diff{}{ \grad{ F } }( v ) \times v }  &=
        \skaq{ v } \mult \diff{2}{ F }( v, \grad{ F } )\\
        \skap{ \grad{ F } \times v }{ \diff{}{ \grad{ F } }( v ) \times \dot{ v } }  &=
        ( \diff{2}{ F }( v ) )^2\\
        \skap{ \grad{ F } \times v }{ \diff{}{ \grad{ F } }( \dot{ v } ) \times v }  &=
        -  \skaq{ v } \mult \frac{ \diff{2}{ F }( \grad{ F } ) \mult \diff{2}{ F }( v ) }{ \skaq{ \grad{ F } } }\\
        \skap{ \grad{ F } \times v }{ \diff{2}{ \grad{ F } }( v ) \times v }  &=
        \skaq{ v } \mult \diff{3}{ F }( v, v, \grad{ F } ) .
    \end{align}
\end{subequations}
Equation \eqref{def:dgem:b} yields
\begin{align}
    \skap{ \grad{ F } \times v }{ \delta \dot{ v } }  &=
    -\alpha \mult \frac{ \diff{2}{ F }( \grad{ F } \times v ) \mult \diff{2}{ F }( v ) }{ \skaq{ \grad{ F } } } .
\end{align}
For $ \alpha $ one arrives at the differential equation
\begin{align}
    &\ddot{ \alpha } \mult \skaq{ \grad{ F } } \skaq{ v }  +  \dots\nonumber\\
    &\dot{ \alpha } \mult \diff{2}{ F }( v, \grad{ F } ) \skaq{ v }  +  \dots\nonumber\\
    &\alpha \left[ ( \diff{2}{ F }( v ) )^2  -
    \frac{ \skaq{ v } \diff{2}{ F }( \grad{ F } ) \diff{2}{ F }( v ) }{ \skaq{ \grad{ F } } }  +  \dots\right.\nonumber\\
    &\qquad\left.\skaq{ v } \diff{3}{ F }( v, v, \grad{ F } )  +
    \frac{ \diff{2}{ F }( \grad{ F } \times v ) \diff{2}{ F }( v ) }{ \skaq{ \grad{ F } } } \right]  =  0 .
\end{align}
For $ \alpha $ one arrives at the differential equation
\begin{align}
    &\ddot{ \alpha }  +
    \dot{ \alpha } \mult \frac{ \diff{2}{ F }( v, \grad{ F } ) }
    { \skaq{ \grad{ F } } }  +  \dots\nonumber\\
    &\alpha \left[ \diff{2}{ F }( v ) \left[ \frac{ \diff{2}{ F }( v ) }{ \skaq{ \grad{ F } } \skaq{ v } }  -
    \frac{ \diff{2}{ F }( \grad{ F } ) }{ \skaq{ \grad{ F } }^2 }  +
    \frac{ \diff{2}{ F }( \grad{ F } \times v ) }{ \skaq{ \grad{ F } }^2 \skaq{ v } } \right]  +
    \frac{ \diff{3}{ F }( v, v, \grad{ F } ) }{ \skaq{ \grad{ F } } } \right]  =  0 .
\end{align}

The equation can be cast in the form
\begin{subequations}
    \begin{align}
        \ddot{ \alpha }   =
        c_0 \mult \alpha  +
        c_1 \mult \dot{ \alpha }
    \end{align}
    with the coefficients
    \begin{align}
        &\begin{aligned}
            c_0   &=
            - \frac{ \diff{3}{F}( v, v, \grad{F} ) }{ \skaq{\grad{F}} }  +  \dots \\
            &\hphantom{={}}- \diff{2}{F}( v )
            \left[ \frac{ 1 }{ \skaq{v} }
            \left[ \frac{ \diff{2}{F}( v ) }{ \skaq{\grad{F}} }  +
            \frac{ \diff{2}{F}( \grad{F} \times v ) }{ \skaq{\grad{F}}^2 } \right]  -
            \frac{ \diff{2}{F}( \grad{F} ) }{ \skaq{\grad{F}}^2 } \right]
        \end{aligned}\\
        &c_1   =
        - \frac{ \diff{2}{F}( v, \grad{F} ) }{ \skaq{\grad{F}} }
    \end{align}
\end{subequations}
The coefficients can be rewritten a little more compactly as
\begin{subequations}
    \begin{align}
        &\begin{aligned}
            c_0   &=
            - \diff{3}{F}( v, v, \darg{F} )  +  \dots \\
            &\hphantom{=}-\diff{2}{F}( v )
            \left[ \frac{ 1 }{ \skaq{v} }
            \left[ \frac{ \diff{2}{F}( v ) }{ \skaq{\grad{F}} }  +
            \diff{2}{F}( \darg{F} \times v ) \right]  -
            \diff{2}{F}( \darg{F} ) \right] ,
        \end{aligned}\\
        &c_1   =   - \diff{2}{F}( v, \darg{F} ) .
    \end{align}
\end{subequations}




Summing up, the calculations have shown
that when $ u_0 $ is parallel to $ \grad{F} \times v $
then $ \operatorname{Exp}( F, p_0, v_0, u_0 ) $ can be calculated
via the following time evolution problem
\begin{align}
    \alpha( 0 )  =  0  \\
    \dot{ \alpha }( 0 )  =  \frac{ \abs{ u_0 } }{ \abs{ \grad{F}( 0 ) } }
\end{align}
mit
\begin{align}
    \ddot{ \alpha }   =
    C \mult \alpha  +  D \mult \dot{ \alpha }
\end{align}
and the evolution equation
\begin{subequations}
    \begin{align}
        \delta \dot{ v }   &=
        \darg{F}( v ) \mult
        (   \skap{ C_1( v ) }{ \delta p }  + 
            \skap{ C_2( v ) }{ \delta v }     )  +
        C_3( v, \delta p )
    \end{align}
    with
    \begin{align}
        C_1( v )  &=
        -\diff{2}{\grad{F}}( v )  +
        2 \mult \diff{}{\grad{F}}( \darg{F} ) \mult \diff{2}{F}( v )  \\
        C_2( v )  &=
        -2 \mult \diff{}{\grad{F}}( v )  \\
        C_3( v, \delta p )  &=
        - \diff{}{\grad{F}}( \delta p ) \mult \frac{ \diff{2}{F}(v ) }{ \skaq{ \grad{F} } }
    \end{align}
\end{subequations}



Transversals property
\begin{align}
    \skap{ \delta p }{ v }^\bullet  =
    \skap{ \delta v }{ v }  +  \skap{ \delta p }{ \dot{ v } }  =
    \delta( \tfrac12 \skaq{ v } )  +  \skap{ \delta p }{ \dot{ v } }
\end{align}