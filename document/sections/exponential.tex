\section{Exponential Mapping}

The exponential mapping associated to $ F $ is defined as
\begin{subequations}
    \label{def:gem}
    \begin{align}
        \label{def:gem:a}
        \operatorname{Exp}( F; p_0; v_0 )  :=
        p_0  +  \int^1_0 v( t ) \intd{ t } ,
    \end{align}
    where $ v $ is a geodesic satisfying the initial conditions
    \begin{align}
        p( 0 )  &\overset{ ! }{ = }   p_0 ,
        \label{def:gem:b}\\
        v( 0 )  &\overset{ ! }{ = }   v_0 ,
        \label{def:gem:c}
    \end{align}
\end{subequations}
where $ p_0 $ is a point on $ M $ and
$ v_0 $ is a vector tangential to $ M $ at $ p_0 $.
Thus, the value $ \operatorname{Exp}(F;p_0;v_0) $
is determined by the Equations \eqref{def:cf},
\eqref{def:gem:b} and \eqref{def:gem:c}.


The partial derivative of the exponential mapping
in the variable $ v_0 $ is explained by
\begin{subequations}
    \label{def:dgem}
    \begin{align}
        \diff{}{ \operatorname{Exp} }( F; p_0; v_0; u_0 )  =
        \int^1_0 \delta v( t ) \intd{ t }
        \label{def:dgem:a}
    \end{align}
    where
    \begin{align}
        \delta \dot{ v }   &=
        A( v, \delta p )  +  B( v, \delta v )
        \label{def:dgem:b}\\
        \delta p( 0 )  &\overset{ ! }{ = }  0,
        \label{def:dgem:c}\\
        \delta v( 0 )  &\overset{ ! }{ = }  u_0
        \label{def:dgem:d}
    \end{align}
    and
    \begin{align}
        &\begin{aligned}
            A( v, \delta p )   &=
            - \frac{ \grad{F} }{ \skaq{ \grad{F} } }
            \left(  \diff{3}{F}( v, v, \delta p )  -
            2 \mult \diff{2}{F}( \delta p, \grad{F} ) \mult
            \frac{ \diff{2}{F}( v ) }{ \skaq{\grad{F}} }  \right)  +  \dots\\
            &\hphantom{={}} - \diff{}{\grad{F}}( \delta p ) \mult
            \frac{ \diff{2}{F}( v ) }{ \skaq{\grad{F}} } ,
        \end{aligned}
        \label{def:dgem:e}\\
        &B( v, \delta v )   =
        - 2 \mult \frac{ \grad{F} }{ \skaq{ \grad{F} } }  \mult
        \diff{2}{F}( v, \delta v ) .
        \label{def:dgem:f}
    \end{align}
\end{subequations}



Notizen zu den Berechnungen
\begin{subequations}
    \begin{align}
    \delta \left( \frac{ \grad{ F } }{ \skaq{ \grad{ F } } } \right)  &=
    \frac{ \diff{}{ \grad{ F } }( \delta p ) }{ \skaq{ \grad{ F } } }  -
    2 \mult \grad{ F } \mult \frac{ \diff{ 2 }{ F }( \delta p, \grad{ F } ) }{ \skaq{ \grad{ F } }^2 }
    \label{vvdgl}\\
    \delta\left( \diff{ 2 }{ F }( v ) \right)  &=
    \diff{ 3 }{ F }( v, v, \delta p )  +
    2 \mult \diff{ 2 }{ F }( v, \delta v )
    \end{align}
\end{subequations}


\subsection{Transversal distortion factors}


Transversals property
\begin{align}
    \skap{ \delta p }{ v }^\bullet  =
    \skap{ \delta v }{ v }  +  \skap{ \delta p }{ \dot{ v } }  =
    \delta( \tfrac12 \skaq{ v } )  +  \skap{ \delta p }{ \dot{ v } }
    \label{def:ta}
\end{align}
Consider the Ansatz for the transversal case
\begin{align}
    \delta p  =  \alpha \mult \grad{ F } \times v .
    \label{eq:tad}
\end{align}
This implies that
\begin{align}
    \delta v  =
    \dot{ \alpha } \mult \grad{ F } \times v  +
    \alpha \mult \diff{}{ \grad{ F } }( v ) \times v
    \label{eq:tadd}
\end{align}
and
\begin{align}
    \delta \dot{ v }  &=
    \ddot{ \alpha } \mult \grad{ F } \times v  +  \dots\nonumber\\
    &\hphantom{{}={}}\dot{ \alpha } \mult 2 \diff{}{ \grad{ F } }( v ) \times v  +  \dots\nonumber\\
    &\hphantom{{}={}}\alpha \mult \left[ \diff{}{ \grad{ F } }( v ) \times \dot{ v }  +
    \diff{}{ \grad{ F } }( \dot{ v } ) \times v  +  
    \diff{ 2 }{ \grad{ F } }( v ) \times v \right] .
\end{align}
Note, that the initial condition \eqref{def:dgem:c} translates to
\begin{align}
    \alpha( 0 )  \overset{!}{=}  0
\end{align}
and the initial condition \eqref{def:dgem:c} and \eqref{def:dgem:d} yield
\begin{align}
    \delta v( 0 )   \overset{!}{=}
    \dot{ \alpha }( 0 ) \mult \grad{F}( p( 0 ) ) \times v( 0 )  =  u_0 ,
\end{align}
where $ u_0 $ is assumed to be transversal.
Then it follows 
\begin{align}
    \dot{ \alpha }( 0 )   \overset{!}{=}
    \frac{ \abs{ u_0 } }{ \abs{ \grad{F}[ p_0 ] } } .
\end{align}
One calculates the expressions
\begin{subequations}
    \begin{align}
        \skap{ \grad{ F } \times v }{ \grad{ F } \times v }  &=
        \skaq{ \grad{ F } } \skaq{ v }\\
        \skap{ \grad{ F } \times v }{ \diff{}{ \grad{ F } }( v ) \times v }  &=
        \skaq{ v } \mult \diff{2}{ F }( v, \grad{ F } )\\
        \skap{ \grad{ F } \times v }{ \diff{}{ \grad{ F } }( v ) \times \dot{ v } }  &=
        ( \diff{2}{ F }( v ) )^2\\
        \skap{ \grad{ F } \times v }{ \diff{}{ \grad{ F } }( \dot{ v } ) \times v }  &=
        -  \skaq{ v } \mult \frac{ \diff{2}{ F }( \grad{ F } ) \mult \diff{2}{ F }( v ) }{ \skaq{ \grad{ F } } }\\
        \skap{ \grad{ F } \times v }{ \diff{2}{ \grad{ F } }( v ) \times v }  &=
        \skaq{ v } \mult \diff{3}{ F }( v, v, \grad{ F } ) .
    \end{align}
\end{subequations}
Equation \eqref{vvdgl} yields
\begin{align}
    \skap{ \grad{ F } \times v }{ \delta \dot{ v } }  &=
    -\alpha \mult \frac{ \diff{2}{ F }( \grad{ F } \times v ) \mult \diff{2}{ F }( v ) }{ \skaq{ \grad{ F } } } .
\end{align}
For $ \alpha $ one arrives at the differential equation
\begin{align}
    &\ddot{ \alpha } \mult \skaq{ \grad{ F } } \skaq{ v }  +  \dots\nonumber\\
    &\dot{ \alpha } \mult \diff{2}{ F }( v, \grad{ F } ) \skaq{ v }  +  \dots\nonumber\\
    &\alpha \left[ ( \diff{2}{ F }( v ) )^2  -
    \frac{ \skaq{ v } \diff{2}{ F }( \grad{ F } ) \diff{2}{ F }( v ) }{ \skaq{ \grad{ F } } }  +  \dots\right.\nonumber\\
    &\qquad\left.\skaq{ v } \diff{3}{ F }( v, v, \grad{ F } )  +
    \frac{ \diff{2}{ F }( \grad{ F } \times v ) \diff{2}{ F }( v ) }{ \skaq{ \grad{ F } } } \right]  =  0 .
\end{align}
For $ \alpha $ one arrives at the differential equation
\begin{align}
    &\ddot{ \alpha }  +
    \dot{ \alpha } \mult \frac{ \diff{2}{ F }( v, \grad{ F } ) }
    { \skaq{ \grad{ F } } }  +  \dots\nonumber\\
    &\alpha \left[ \diff{2}{ F }( v ) \left[ \frac{ \diff{2}{ F }( v ) }{ \skaq{ \grad{ F } } \skaq{ v } }  -
    \frac{ \diff{2}{ F }( \grad{ F } ) }{ \skaq{ \grad{ F } }^2 }  +
    \frac{ \diff{2}{ F }( \grad{ F } \times v ) }{ \skaq{ \grad{ F } }^2 \skaq{ v } } \right]  +
    \frac{ \diff{3}{ F }( v, v, \grad{ F } ) }{ \skaq{ \grad{ F } } } \right]  =  0 .
\end{align}
Written differently:
\begin{subequations}
    \begin{align}
        \ddot{ \alpha }   =   \dot{ \alpha } \mult C  +  \alpha \mult D
    \end{align}
    with
    \begin{align}
        &\begin{aligned}
            C   &=
            - \frac{ \diff{3}{F}( v, v, \grad{F} ) }{ \skaq{\grad{F}} }  +  \dots \\
            &- \diff{2}{F}( v )
            \left[ \frac{ 1 }{ \skaq{v} }
            \left[ \frac{ \diff{2}{F}( v ) }{ \skaq{\grad{F}} }  +
            \frac{ \diff{2}{F}( \grad{F} \times v ) }{ \skaq{\grad{F}}^2 } \right]  -
            \frac{ \diff{2}{F}( \grad{F} ) }{ \skaq{\grad{F}}^2 } \right]
        \end{aligned}\\
        &D   =   - \frac{ \diff{2}{F}( v, \grad{F} ) }{ \skaq{\grad{F}} }
    \end{align}
\end{subequations}
or, written differently as
\begin{subequations}
    \begin{align}
        &\begin{aligned}
            C   &=
            - \diff{3}{F}( v, v, \darg{F} )  +  \dots \\
            &\hphantom{=}-\diff{2}{F}( v )
            \left[ \frac{ 1 }{ \skaq{v} }
            \left[ \frac{ \diff{2}{F}( v ) }{ \skaq{\grad{F}} }  +
            \diff{2}{F}( \grad{F} \times v ) \right]  -
            \diff{2}{F}( \darg{F} ) \right]
        \end{aligned}\\
        &D   =   - \diff{2}{F}( v, \darg{F} )
    \end{align}
    % where
    % \begin{align}
    %     v^\circ   :=
    %     \darg{F} \times v
    % \end{align}
    % und thus
    % \begin{align}
    %     \delta p   =
    %     \frac{ \alpha }{ \skaq{\grad{F}} } \mult v^\circ
    % \end{align}
\end{subequations}



It still needs to be verified, that the transversal Ansatz works out.
Therefore one calculates
\begin{align}
    \skap{ v }{ \delta \dot{ v } }  &\overset{\eqref{def:dgem:b}}{=}
    -\alpha \mult \skap{ v }{ \diff{}{\grad{F}}( \grad{F} \times v ) }  \mult
    \frac{ \diff{2}{F}( v ) }{ \skaq{ \grad{F} } }
\end{align}
Heyho, nicht so viel Whiskey trinken:
\begin{align}
    \skap{ v }{ \delta \dot{ v } }  &\overset{\eqref{eq:tadd}}{=}
    -\alpha \mult \skap{ v }{ \diff{}{\grad{F}}( v ) \times \grad{F} }  \mult
    \frac{ \diff{2}{F}( v ) }{ \skaq{ \grad{ F } } }
\end{align}
Hierbei berechnet man
\begin{align}
    \skap{ v }{ \diff{}{\grad{F}}( v ) \times \grad{F} }  =
    \skap{ v }{ \diff{}{\grad{F}}( \grad{F} \times v ) }  =
    \diff{2}{F}( v, \grad{F} \times v )
\end{align}
und somit ist die $ v $-Komponente der DGL \eqref{def:dgem:b} eine Tautologie.


Die $ \grad{F} $-Komponente:
\begin{subequations}
    \label{vdgk:1}
\begin{align}
    \skap{ \grad{F} }{ - \darg{F} \mult
    \diff{3}{F}( v, v, \delta p ) }   =
    - \alpha \mult \diff{3}{F}( v, v, \grad{F} \times v )
    \label{vdgk:1:a}
\end{align}
dann:
\begin{align}
    \skap{ \grad{F} }{ -2 \darg{F} \mult \diff{2}{F}( v, \delta p ) }   &=
    - 2 \dot{ \alpha }  \mult  \diff{2}{F}( v, \grad{F} \times v )  -
    2 \alpha  \mult \diff{2}{F}( v, \diff{}{\grad{F}}( v ) \times v )  \nonumber\\
    &=   - 2 \dot{ \alpha }  \mult  \diff{2}{F}( v, \grad{F} \times v )
    \label{vdgk:1:b}
\end{align}
wegen
\begin{align*}
    \diff{2}{F}( v, \diff{}{\grad{F}}( v ) \times v )  =
    \skap{ \diff{}{\grad{F}}( v ) }{ \diff{}{\grad{F}}( v ) \times v }  =  0
\end{align*}
dann:
\begin{align}
    \skap{ \grad{F} }
    { 2 \mult \darg{F} \mult \diff{2}{F}( v ) \mult
    \diff{2}{F}( \delta p, \darg{F} ) }   &=
    2 \alpha \mult \diff{2}{F}( v ) \mult
    \diff{2}{F}( \grad{F}, \darg{F} \times v )
    \label{vdgk:1:c}
\end{align}
weiter:
\begin{align}
    \skap{ \grad{F} }
    { -\diff{}{\grad{F}}( \delta p ) \mult \diff{2}{F}( v ) / \skaq{ \grad{F} } }
    =   - \alpha \mult \diff{2}{F}( v ) \mult
    \diff{2}{F}( \grad{F}, \darg{F} \times v )
    \label{vdgk:1:d}
\end{align}
\end{subequations}
Andererseits:
\begin{subequations}
    \label{vdgk:2}
\begin{align}
    \skap{ \grad{F} }
    { 2 \dot{ \alpha } \mult \diff{}{\grad{F}}( v ) \times v }   =
    - 2 \dot{ \alpha } \mult
    \skap{ \grad{F} \times v }{ \diff{}{ \grad{F}( v ) } }   =
    - 2 \dot{ \alpha } \mult \diff{2}{F}( v, \grad{F} \times v )
    \label{vdgk:2:a}
\end{align}

\begin{align}
    \skap{ \grad{F} }
    { \alpha \mult \diff{}{ \grad{F} }( \dot{ v } ) \times v }   &=
    \alpha \mult \diff{2}{F}( v ) \mult
    \diff{2}{F}( \grad{F}, \darg{F} \times v )
    \label{vdgk:2:b}
\end{align}

\begin{align}
    \skap{ \grad{F} }
    { \alpha \mult \diff{2}{\grad{F}}( v ) \times v }   &=
    - \alpha \mult \diff{3}{F}( v, v, \grad{F} \times v )
    \label{vdgk:2:c}
\end{align}
\end{subequations}
Nun sieht man, dass $ \eqref{vdgk:1:a}  =  \eqref{vdgk:2:c} $,
$ \eqref{vdgk:1:b}  =  \eqref{vdgk:2:a} $ und
$ \eqref{vdgk:1:c} + \eqref{vdgk:1:d}  =  \eqref{vdgk:2:b} $.
Also ist die $ \grad{F} $-Komponente der Variationsdgl immer erfüllt.


Insgesamt ist damit gezeigt,
dass das Anfangsproblem \eqref{} bereits die Entwicklung
von $ \delta p  =  \alpha \mult \grad{F} \times v $ korrekt beschreibt.