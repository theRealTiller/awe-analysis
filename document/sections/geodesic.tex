\section{Geodesics}


Notations for the level function:
\begin{subequations}
\begin{align}
\mathbb{ E }  &:  \text{Euclidean space}\\
p  &:  \text{position in Euclidean space}\\
v  &:  \text{tangent vector at $ p $}\\
F( p )  &:  \text{level function at position $ p $}\\
\diff{}{ F }( p; v )  &:  \text{derivative of the level function at position $ p $ in direction $ v $}\\
\grad{ F }( p )  &:  \text{gradient of the level function at position $ p $}
\end{align}
Surface definition:
\begin{align}
\mathbb{ S }  &:=  \{ p \in \mathbb{ E } \,;\, F( p )  = 0 \}
\end{align}
\end{subequations}



\subsection{Basic differential equation for geodesics}



Variables for geodesic:
\begin{subequations}
\begin{align}
t  &:  \text{time}\\
p( t )  &:  \text{position at time $ t $}\\
v( t )  &:  \text{velocity at time $ t $}
\end{align}
\end{subequations}
Simplified notation:
\begin{subequations}
\begin{align}
p  &\equiv  p( t )\\
v  &\equiv  v( t )
\end{align}
\end{subequations}
Geodesic equations:
\begin{subequations}
\label{eq:geo}
\begin{align}
F( p )  &=  0
\label{eq:geo:a}\\
\dot{ v }  &\para  \grad{ F }( p )
\label{eq:geo:b}
\end{align}
\end{subequations}
Simplified notation:
\begin{subequations}
\begin{align}
\diff{}{ F }( v )  &\equiv  \diff{}{ F }( p; v )\\
\grad{ F }         &\equiv  \grad{ F }( p )
\end{align}
\end{subequations}
Conclusion from \eqref{eq:geo:a}:
\begin{align}
\diff{}{ F }( p; v )  &=  0
\label{eq:geo:aa}
\end{align}
This is equivalent to
\begin{align}
\skap{ \grad{ F }( p ) }{ v }  =  0
\end{align}

As a conclusion from \eqref{eq:geo:aa} and \eqref{eq:geo:b}
one obtains the differential equation that describes the geodesic motion:
\begin{align}
\dot{ v }   &=  -\frac{ \grad{ F } }{ \skaq{ \grad{ F } } } \mult \diff{ 2 }{ F }( v )
\end{align}
It follows that Geodesics have constant speed because
\begin{align}
    \skap{ v }{ v }^\bullet  =  2 \skap{ v }{ \dot{ v } }  =  0 .
\end{align}


\subsection{Higher derivatives}


Some preliminary calculations:
\begin{subequations}    
    \begin{align}
        \left( \skaq{ \grad{F} }^\alpha \right)^\bullet
        &=
        2 \alpha \mult \skaq{ \grad{F} }^{\alpha-1} \mult
        \skap{ \diff{}{F}( v ) }{ \grad{F} }   \\
        &=
        2 \alpha \mult \skaq{ \grad{F} }^{\alpha-1} \mult
        \diff{2}{F}( v, \grad{F} )
    \end{align}
\end{subequations}
Further preliminary calculations:
\begin{align}
\left( \frac{ \grad{ F } }{ \skaq{ \grad{ F } } } \right)^\bullet
&=   \frac{ \diff{}{ \grad{ F } }( v ) }{ \skaq{ \grad{ F } } }
            -  2 \mult \grad{ F } \mult \frac{ \diff{ 2 }{ F }( v, \grad{ F } ) }{ \skaq{ \grad{ F } }^2 }\\
\left( \diff{ 2 }{ F }( v ) \right)^\bullet  &=
\diff{ 3 }{ F }( v )  -  2 \mult \frac{ \diff{ 2 }{ F }( v, \grad{ F } ) }{ \skaq{ \grad{ F } } } \mult \diff{2}{F}( v )
\end{align}
Conclusion from \eqref{eq:geo:aa}:
\begin{subequations}
\begin{align}
\ddot{ v }   &=
- \frac{ \grad F \mult A( v )  +
\diff{}{ \grad{ F } }( v ) \mult \diff{ 2 }{ F }( v ) }{ \skaq{ \grad{ F } } }
\end{align}
Used abbreviation:
% \begin{align}
% A( v )  &:=
% \diff{ 3 }{ F }( v ) -
% 2 \mult \left( 1 + \diff{ 2 }{ F }( v ) \right) \mult
% \frac{ \diff{ 2 }{ F }( v, \grad{ F } ) }{ \skaq{ \grad{ F } } }
% \end{align}
\begin{align}
    A( v )  &:=
    \diff{ 3 }{ F }( v ) -
    4 \mult \diff{ 2 }{ F }( v ) \mult
    \frac{ \diff{ 2 }{ F }( v, \grad{ F } ) }{ \skaq{ \grad{ F } } }
\end{align}
\end{subequations}