\documentclass{scrartcl}


\usepackage[utf8]{inputenc}
\usepackage{ngerman}
\usepackage[T1]{fontenc}

\usepackage{amsmath,amssymb}


\newcommand{ \skap }[ 2 ]{ \langle #1, #2 \rangle }
\newcommand{ \Skap }[ 2 ]{ \left\langle #1, #2 \right\rangle }

\newcommand{ \skaq }[ 1 ]{ \langle #1 \rangle }
\newcommand{ \Skaq }[ 1 ]{ \left\langle #1 \right\rangle }

\newcommand{ \abs }[ 1 ]{ \lvert #1 \rvert }

\newcommand{ \ormi }{ \mathbin{ \dashv } }

\newcommand{ \normed }[ 1 ]{ \widehat{ #1 } }

\newcommand{ \para }{ \parallel }

\newcommand{ \mult }{ \cdot }

\newcommand{ \diff }[ 2 ]{ \mathrm{D}^{ #1 }#2 }
\newcommand{ \grad }[ 1 ]{ \nabla#1 }

\newcommand{ \intd }[ 1 ]{ \,\mathrm{ d }{ #1 } }


\begin{document}


\newpage
\section{Geodesics}


Variables:
\begin{subequations}
\begin{align}
t  &:  \text{time}\\
p( t )  &:  \text{position at time $ t $}\\
v( t )  &:  \text{velocity at time $ t $}\\
F( p )  &:  \text{level function at position $ p $}
\end{align}
\end{subequations}
Simplified notation:
\begin{subequations}
\begin{align}
p  &\equiv  p( t )\\
v  &\equiv  v( t )
\end{align}
\end{subequations}
Geodesic equations:
\begin{subequations}
\label{eq:geo}
\begin{align}
F( p )  &=  0
\label{eq:geo:a}\\
\dot{ v }  &\para  \grad{ F }( p )
\label{eq:geo:b}
\end{align}
\end{subequations}
Simplified notation:
\begin{subequations}
\begin{align}
\diff{}{ F }( v )  &\equiv  \diff{}{ F }( p; v )\\
\grad{ F }         &\equiv  \grad{ F }( p )
\end{align}
\end{subequations}
Conclusion from \eqref{eq:geo:a}:
\begin{align}
\diff{}{ F }( p; v )  &=  0
\label{eq:geo:aa}
\end{align}
Conclusion from \eqref{eq:geo:aa} and \eqref{eq:geo:b}:
\begin{align}
\dot{ v }   &=  -\frac{ \grad{ F } }{ \skaq{ \grad{ F } } } \mult \diff{ 2 }{ F }( v )
\end{align}
Some preliminary calculations:
\begin{align}
\left( \frac{ \grad{ F } }{ \skaq{ \grad{ F } } } \right)^\bullet
&=   \frac{ \diff{}{ \grad{ F } }( v ) }{ \skaq{ \grad{ F } } }
            -  2 \mult \grad{ F } \mult \frac{ \diff{ 2 }{ F }( v, \grad{ F } ) }{ \skaq{ \grad{ F } }^2 }\\
\left( \diff{ 2 }{ F }( v ) \right)^\bullet  &=
\diff{ 3 }{ F }( v )  -  2 \mult \frac{ \diff{ 2 }{ F }( v, \grad{ F } ) }{ \skaq{ \grad{ F } } }
\end{align}
Conclusion from \eqref{eq:geo:aa}:
\begin{subequations}
\begin{align}
\ddot{ v }   &=
- \frac{ \grad F \mult A( v )  +
\diff{}{ \grad{ F } }( v ) \mult \diff{ 2 }{ F }( v ) }{ \skaq{ \grad{ F } } }
\end{align}
Used abbreviation:
\begin{align}
A( v )  &:=
\diff{ 3 }{ F }( v ) -
2 \mult \left( 1 + \diff{ 2 }{ F }( v ) \right) \mult
\frac{ \diff{ 2 }{ F }( v, \grad{ F } ) }{ \skaq{ \grad{ F } } }
\end{align}
\end{subequations}


\newpage
\section{Curvature}


Geometric quantities:
\begin{subequations}
\begin{align}
\kappa( t )  &:  \text{curvature at time $ t $}\\
\tau( t )    &:  \text{torsion at time $ t $}
\end{align}
\end{subequations}
Simplified notation:
\begin{subequations}
\begin{align}
\kappa  \equiv  \kappa( t )\\
\tau    \equiv  \tau( t )
\end{align}
\end{subequations}
Geometric equations:
\begin{subequations}
\begin{align}
\kappa   &=   -\frac{ \diff{ 2 }{ F }( v ) }{ \abs{ \grad{ F } } \mult \skaq{ v } }\\
\tau     &=   -\frac{ \diff{ 2 }{ F }( v, v \times \grad{ F } ) }{ \skaq{ \grad{ F } } \mult \skaq{ v } }
\end{align}
\end{subequations}


\newpage
\section{Derivative of the Exponential Mapping}


\begin{subequations}
\begin{align}
\exp  &:  \text{exponential mapping}\\
p_0  &:  \text{initial position}\\
v_0  &:  \text{initial velocity}
\end{align}
\end{subequations}


\begin{subequations}
\begin{align}
\exp( p_0; v_0 )  &=  p_0  +  \int^1_0 v( t ) \intd{ t }\\
p( 0 )  &\overset{ ! }{ = }   p_0\\
v( 0 )  &\overset{ ! }{ = }   v_0
\end{align}
\end{subequations}


%% \begin{align}
%% \exp( p_0; v_0 )  &=  p_0  +  t \mult v_0  +  \int^1_0\left( \int^t_0 \dot{ v }( s ) \intd{ s } \right) \intd{ t }
%% \end{align}


\begin{subequations}
\begin{align}
\delta \left( \frac{ \grad{ F } }{ \skaq{ \grad{ F } } } \right)  &=
\frac{ \diff{}{ \grad{ F } }( \delta p ) }{ \skaq{ \grad{ F } } }  -
2 \mult \grad{ F } \mult \frac{ \diff{ 2 }{ F }( \delta p, \grad{ F } ) }{ \skaq{ \grad{ F } }^2 }\\
\delta\left( \diff{ 2 }{ F }( v ) \right)  &=
\diff{ 3 }{ F }( v, v, \delta p )  -  2 \mult \diff{ 2 }{ F }( v, \delta v )
\end{align}
\end{subequations}


\begin{subequations}
\begin{align}
\delta \dot{ v }   &=
- \frac{ \grad F \mult A( v, \delta p, \delta v  )
+  \diff{}{ \grad{ F } }( \delta p ) \mult \diff{ 2 }{ F }( v ) }{ \skaq{ \grad{ F } } } \\
\delta p( 0 )  &\overset{ ! }{ = }  0,\\
\delta v( 0 )  &\overset{ ! }{ = }  u.
\end{align}

\begin{align}
A( v, \delta p, \delta v  )  &:=
\diff{ 3 }{ F }( v, v, \delta p ) -
         2 \mult \left( \diff{ 2 }{ F }( v, \delta v ) +
                        \frac{ \diff{ 2 }{ F }( v ) \mult \diff{ 2 }{ F }( \delta p, \grad{ F } )
                                       }{ \skaq{ \grad{ F } } } \right)
\end{align}
\end{subequations}



\begin{align}
\diff{}{ \exp}( p_0; v_0, u )  =  \int^1_0 \delta v( t ) \intd{ t }
\end{align}


\end{document}
